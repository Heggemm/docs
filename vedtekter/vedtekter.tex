\documentclass[8pt,norsk,a4paper]{article}
\usepackage[norsk]{babel}
\usepackage[utf8]{inputenc}
\usepackage[T1]{fontenc}
\usepackage{parskip,textcomp,amssymb,fullpage}
\usepackage[margin=2.5cm,headsep=1cm]{geometry}
\usepackage[bitstream-charter]{mathdesign}

% Fractions
\usepackage{xfrac}
\DeclareInstance{xfrac}{default}{text}%
{%
	scale-factor = 1.2,
	numerator-bot-sep = 3pt,
	denominator-bot-sep = -1pt,
}

% Redefines sections in the format §1 ...
\renewcommand{\thesection}{§\arabic{section}}
% Redefines the first level of enumerated lists.
\renewcommand{\theenumi}{\alph{enumi})}
\renewcommand{\labelenumi}{\theenumi}

\title{\textbf{Cybernetisk Selskabs vedtekter} \\
	{\large Oppdatert etter generalforsamlingen 10.11.2016*\thanks{* Oppdatert med rettskriving april 2017.}.}}
\date{}
\author{}


\begin{document}
\maketitle{}

\section{Tilhørighet og formål}
\begin{enumerate}
	\item{Cybernetisk Selskab er en instituttforening ved Institutt for informatikk ved Universitetet i Oslo.}
	\item{Cybernetisk Selskabs formål er å arrangere og fremme aktiviteter og arrangementer for studenter ved instituttet. Gjennom dette ønsker man å styrke miljøet og skape kameratslig samvær og faglig innhold ved siden av studiene.}
\end{enumerate}

\section{Medlemskap}\label{sec:medlemskap}
\begin{enumerate}
	\item{Cybernetisk Selskab har tre kategorier medlemmer:}
	\begin{itemize}
		\item{Semesterbetalende medlemmer.}
		\item{Livsvarige medlemmer.}
		\item{Æresmedlemmer.}
	\end{itemize}
	\item{Kontingent for livsvarlig medlemskap er ti ganger ordinær semesterkontingent.}
	\item{Æresmedlemmer utnevnes på generalforsamling med \sfrac{2}{3} flertall.}
\end{enumerate}

\section{Drift og arbeidsgrupper}
\begin{enumerate}
	\item{Foreningens daglige drift ivaretas av arbeidsgrupper med hvert sitt ansvarsområde. Arbeidsgruppene skal i størst mulig grad handle fritt og selvstendig innenfor de rammer som er satt av hovedstyret.}
	\item{Hovedstyret har rett til innsyn i understyrer og arbeidsgruppers bruk av midler.}
	\item{Understyrer og arbeidsgrupper plikter å rapportere regnskapsrelevant informasjon til kasserer og overholde vedtatte budsjetter.}
	\item{Om nødvendig kapital til foreningens drift (100~000,- kroner) samt bardrift (400~000,- kroner) og vedlikehold (250~000,- kroner) er oppspart skal et eventuelt driftoverskudd overføres Fordelingsutvalget ved årsskifte.}
\end{enumerate}

\section{Revisjon}
\begin{enumerate}
	\item{Hovedstyret skal utnevne minst to personer for å revidere årsregnskapet. Personene kan ikke ha vært medlemmer av økonomigruppa eller innehatt styreverv i perioden som revideres.}
	\item{Etter at revideringsrapporten er ferdigstilt, skal den presenteres på første mulige generalforsamlig.}
\end{enumerate}

\section{Hovedstyret}
\begin{enumerate}
	\item{Hovedstyret er med \sfrac{3}{4} flertall av hovedstyret vedtaksdyktige.}
	\item{Hovedstyret skal drive Cybernetisk Selskabs virksomhet, og sikre aktivitet og gjennomføring av foreningens formål.}
	\item{Hovedstyret i Cybernetisk Selskab skal fungere som et sentralt hovedstyre for sine understyrer og arbeidsgrupper.}
	\item{Hovedstyret skal bestå av fem til ti personer, hvor vervene leder, nestleder, kasserer, kjellermogul og arrangementssjef er faste. Øvrige verv defineres av hovedstyret før generalforsamling.}
	\item{Hovedstyret konstituerer seg i samråd med lederen på periodens første hovedstyremøte som skal skje senest to måneder etter generalforsamling.}
	\item{Hovedstyret kan med \sfrac{3}{4} flertall oppnevne hovedstyremedlemmer ved behov for etterfylling av alle verv foruten leder og kjellermogul. Den etterfylte sitter frem til neste generalforsamling.} 
	\item{Alle medlemmer jf.~\ref{sec:medlemskap} har møterett på hovedstyrets møter, men kun styrets medlemmer har stemmerett.}
\end{enumerate}

\section{Kjellerstyret}
\begin{enumerate}
	\item{Kjellerstyret har ansvaret for den daglige driften av kjelleren i Ole-Johan Dahls hus.}
	\item{Kjellerstyret skal ha fra fem til ti personer, og ledes av kjellermogul med barsjef som stedfortreder.}
	\item{Faste verv i kjellerstyret er kjellermogul, barsjef og økonomiansvarlig. Øvrige verv defineres av kjellerstyret før generalforsamling.}
	\item{Kjellerstyret har ansvar for å rapportere regnskapsrelevant informasjon til kasserer og overholde vedtatte budsjetter.}
	\item{Hovedstyret kan oppnevne kjellerstyremedlemmer ved behov for etterfylling av alle verv foruten kjellermogul.}
 	\item{Alle medlemmer jf.~\ref{sec:medlemskap} har møterett på kjellerstyrets møter, men kun styrets medlemmer har stemmerett.}
\end{enumerate}

\section{Generalforsamling}
\begin{enumerate}
	\item{Generalforsamlingen er foreningens høyeste myndighet. Den er beslutningsdyktig når minst 10 prosent eller 30 stykker av medlemmene er tilstedet.}
	\item{Hovedstyret innkaller til generalforsamling. Innkallingen skal komme minimum to uker før generalforsamlingen finner sted for ordinær generalforsamling. Minimum en uke før for ekstraordinær generalforsamling. Foreløpig dagsorden offentliggjøres minst én uke i forkant til ordinær generalforsamling. For ekstraordinær generalforsamling minst tre dager i forveien.}
	\item{Forslag om vedtektsendring og andre saker som søkes tatt opp på generalforsamling må være hovedstyret i hende senest 48 timer i forveien; for ekstraordinær generalforsamling 24 timer i forveien.}
	\item{Generalforsamlingen kan foreta endringer i rekkefølgen av punktene i det endelige forslag til dagsorden. Den kan også utelukke ett eller flere av de foreslåtte punkter, så lenge dette ikke strider mot~\ref{sec:mistillit} og~\ref{sec:opplosning}. Den endelige dagsorden godkjennes av generalforsamlingen.}
	\item{Ethvert medlem kan på generalforsamlingen foreslå tatt opp saker utenom den oppsatte dagsorden. Generalforsamlingen kan ikke fatte vedtak i slike saker.}
	\item{Valg av styremedlemmer på generalforsamling foregår skriftlig dersom det er to eller flere kandidater som stiller til vervet. Øvrige valg foregår ved håndsopprekning med mindre to av de stemmeberettigede ønsker skriftlig avstemming.}
	\item{Stemmerett har alle som er medlem, jf.~\ref{sec:medlemskap}, minst én uke før generalforsamling.}
	\item{Ordinær generalforsamling avholdes i slutten av hvert semester. Ekstraordinær generalforsamling avholdes når hovedstyret, eller minst \sfrac{1}{10} av medlemmene ønsker det.}
	\item{På ordinær generalforsamling skal følgende behandles:}
	\begin{itemize}
		\item{Semesterberetning.}
		\item{Regnskap.}
		\item{Budsjett.}
		\item{Kontingentfastsettelse.}
		\item{Valg.}
	\end{itemize}
	\item{Forslag til vedtektsendring skal behandles på generalforsamling, og må få \sfrac{2}{3} flertall for å bli vedtatt.}
\end{enumerate}

\section{Valg}\label{sec:valg}
\begin{enumerate}
	\item{Valgbare til alle styrer er alle medlemmer jf.~\ref{sec:medlemskap} og informatikkstudenter ved Universitetet i Oslo. Alle valg avgjøres ved simpelt flertall.}
	\item{Hvert hovedstyremedlem velges for to semestre av gangen. Valg foregår på generalforsamling.}
	\item{Alle hovedstyrets medlemmer velges særskilt på generalforsamling.}
	\item{Alle kjellerstyrets medlemmer velges særskilt på generalforsamling. Verv utover kjellermogul og økonomiansvarlig velges for ett semester av gangen. Kjellermogul og økonomiansvarlig velges for to semestre av gangen. }
\end{enumerate}

\section{Mistillit}\label{sec:mistillit}
\begin{enumerate}
	\item{Foreningens medlemmer jf.~\ref{sec:medlemskap} kan fremme mistillitsforslag mot styremedlemmer og andre tillitsvalgte som er valgt i henhold til \ref{sec:valg}.}
	\item{Slike forslag skal behandles på generalforsamling, og må være fremmet minst 48 timer i forveien.}
	\item{Mistillitforslag vedtas med \sfrac{2}{3} flertall.}
	\item{Dersom et mistillitsforslag blir vedtatt kan generalforsamlingen vedta å holde nyvalg for vervet med periode frem til neste ordinære generalforsamling.}
\end{enumerate}

\section{Oppløsning}\label{sec:opplosning}
\begin{enumerate}
        \item{Det kreves \sfrac{2}{3} flertall på to påfølgende generalforsamlinger for å oppløse foreningen.}
        \item{Hvis en generalforsamling vedtar å oppløse foreningen skal hovedstyret tidligst tre uker og senest fire måneder senere innkalle til ekstraordinær generalforsamling hvor saken skal behandles på nytt. Innkallingen skal komme minimum to uker før generalforsamlingen finner sted.}
        \item{Dersom foreningen oppløses, går foreningens midler til Fordelingsutvalget (FU) ved Institutt for informatikk (Ifi).}
\end{enumerate}

\end{document} % chktex 17
